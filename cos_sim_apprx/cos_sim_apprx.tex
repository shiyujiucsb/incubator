\documentclass{article}

\usepackage{graphicx,tikz}
\usepackage{fullpage}
\usepackage{amsthm}
\usepackage{amssymb}
\usepackage{amsmath}
\usepackage{algorithm}
\usepackage{algorithmicx}
\usepackage{algpseudocode}
%\usepackage{booktabs}
%\usepackage{color}

%\renewcommand{\baselinestretch}{0.95}
%\setlength{\textfloatsep}{0.1cm}
%\setlength{\abovecaptionskip}{0.1cm}

\begin{document}

\title{\Large\bf Approximating All-Pairs Similarity Search by Rademacher Average}
\author{Shiyu Ji\\ shiyu@cs.ucsb.edu}
\date{}
\maketitle

\newtheorem{definition}{Definition}
\theoremstyle{definition}
\newtheorem{theorem}{Theorem}
\theoremstyle{plain}
\newtheorem{lemma}{Lemma}
\theoremstyle{plain}
\newtheorem{corollary}{Corollary}
\theoremstyle{plain}

\section{Introduction}
All-pairs similarity search (APSS) has received extensive research interest recently \cite{BMS07,Xia16,ATY13,TAJY14}. To improve performance, many approximation approaches have been proposed \cite{GIM99,FKS03,IM98,Char02}. This paper considers two approximation methods for cosine similarity based search \cite{SGM00,Xia16,ATY13,TAJY14}, and SimRank based search \cite{JW02,LH10,FNS13,KMK14} respectively. The approximation error of each our algorithm is upper bounded by using Rademacher average \cite{BM02,Mohri09,BBM05}.

\section{Problem Formulation and Preliminaries}
\subsection{Cosine Similarity}
We consider the cosine similarity based all-pairs similarity search.
Suppose there are $n$ vectors (each vector can represent a user profile or a web page). Each vector contains $m$ non-negative features. Define the cosine similarity between two vectors $u$ and $v$ as
$$Sim(u,v) = \frac{1}{||u||\cdot||v||} \sum_{i=1}^m u_i\cdot v_i,$$
where $u_i$, $v_i$ denotes the $i$-th feature value of $u$, $v$.
For simplicity, we assume all the vectors are adjusted with the same norm: $||v|| = \sqrt{m}$ for every $v$ in the $n$ vectors. Then the equation above can be simplified as
$$Sim(u,v) = \frac{1}{m} \sum_{i=1}^m u_i\cdot v_i.$$
That is, the similarity is defined as the average on the corresponding feature products between the vectors. 

To do the all-pairs similarity search (APSS), we need to compute the similarity between each pair of vectors. Since there are $n(n-1)$ pairs, and for each pair we need $m$ times of multiplication, the total complexity of a naiive algorithm is $O(n^2 m)$. Fortunately, there are many methods to detect dissimilar pairs (two vectors without sharing any feature) \cite{ATY13,TAJY14,Lin09}, which can save a lot of computation. For the state-of-art works, to compute all the pairs, the complexity can be lowered to $O(nkm)$, where $k$ is much less than $n$. However, to the best of our knowledge, there were few discussions on the size of features $m$. If we can only consider a part of the $m$ features without significantly deteriorating the accuracy, then the total computation time for APSS can be lowered significantly (note that $nk$ is still large for very big dataset). We can approximating the similarity by sampling on the features.

\subsection{SimRank}
SimRank \cite{JW02} is a popular measure of the similarity between two nodes in a graph based on the idea that similar nodes are often referred by other similar nodes. Denote by $s(a,b)$ the SimRank between nodes $a$ and $b$. If $a=b$, then $s(a,b)$ is defined to be 1. Otherwise, 
$$s(a,b) = \frac{c}{|I(a)|\cdot|I(b)|}\sum_{i\in I(a)}\sum_{j\in I(b)}s(i,j),$$
where $c$ is a constant in $(0,1)$ and $I(a)$ denotes the in-neighbors of $a$. If there is an edge from $a$ to $b$, then $a$ is an in-neighbor of $b$. 
An important fact is that SimRank can be equivalently built upon Random Surfer-Pairs Model \cite{JW02}. That is, $s(a,b)$ can also be written as follows:
$$s(a,b) = \sum_{t: (a,b) (x,x)} P[t]\cdot c^{L(t)},$$
where $t$ is a pair of random walks with the same number of steps, which lead $a$ and $b$ to meet at one node $x$, and $P[t]$ denotes the probability when $t$ is chosen, and $L(t)$ denotes the number of steps of each walk in $t$. Here we take random walks from the reverse graph, in which each edge is reversed compared with the original graph. A random walk works as follows: in the tour at any node, which has $k$ out-edges, we take one of the $k$ edge with equal probability $1/k$ as our next step. Note that as the length of random walk grows, its contribution to $s(a,b)$ decreases exponentially. Hence in practice, we only need to consider relatively short walks, e.g., with length no more than $T$.

\subsection{Rademacher Average}
This section proposes our approximation algorithm and its analysis.
Suppose we want to compute the cosine similarities between a fixed vector $u$ and other vectors $v_1$, $v_2$, $\cdots$, $v_n$. We take the $m$ features as the sample space $S$:
$$S = \{s_1, \cdots, s_k\} \subseteq D$$
where $k$ is the number of samples we take, and $D$ is the feature space: $D = \{1,2,\cdots,m\}$.
Let $F$ be a collection of functions from the features $D$ to the interval $[0, m]$.
For each function $f\in F$, define the \emph{true average} $A_D(f)$ and \emph{sampled average} $A_S(f)$ as follows:
$$A_D(f) = \frac{1}{m} \sum_{i=1}^m f(i),\quad A_S(f) = \frac{1}{k} \sum_{i=1}^k f(s_i).$$
Define the \emph{uniform deviation} \cite{Oneto13} of $F$ given $S$ as
$$U_S(F) = \sup_{f\in F} [ A_S(f) - A_D(f) ].$$
Note that if $F$ is a finite set (in this paper we will see this is true), supreme can be replaced by maximum:
$$U_S(F) = \max_{f\in F} [ A_S(f) - A_D(f) ].$$
Define the \emph{Rademacher average} \cite{Mohri09,BM02,Oneto13} of $F$ given $S$ as
\newcommand{\E}{\mathbb{E}}
$$R_S(F) = \E_\sigma \left[\sup_{f\in F} \frac{2}{k}\sum_{i=1}^k \sigma_i f(s_i) \right],$$
where each $\sigma_i$ is a random variable uniformly distributed over $\{-1, 1\}$, and the mean $\E_\sigma$ takes randomness over all the $\sigma_i$'s conditionally on $S$. Again, one can replace supreme by maximum.

The main motivation of our algorithm is that the difference between true average and sampled average is upper bounded by the Rademacher average as follows.
We leave the proofs to the Appendix.

\begin{theorem}
\label{thm:main}
Let $F$ be a collection of functions $f$ mapping $D$ to $[0,m]$. 
With probability at least $1-\delta$, we have
$$\sup_{f\in F}|A_S(f) - A_D(f)| \leq R_S(F) + \left(m+m\sqrt{\frac{8}{k}\log \frac{2}{\delta}} + m\sqrt{\frac{8}{k}\log \frac{2}{\delta} + R_S(F)}\right)\sqrt{\frac{\log \frac{8}{\delta}}{2k}}.$$
\end{theorem}

The remaining item we need to upper bound is $R_S(F)$. Our result is similar to Massart's lemma \cite{AGO14}.

\begin{theorem}
\label{thm2}
$$R_S(F) \leq \frac{\ell}{k}\sqrt{8\log |F|},$$
where $\ell^2 = \sup_{f\in F}\sum_{i=1}^k f(s_i)^2$.
\end{theorem}

By the above two theorems, we can bound our approximation error even for the worst case.


\section{Approximating Cosine Similarities}
The approximation algorithm takes as input a collection of $n$ vectors $V$ and two parameters $(\epsilon, \delta)$ whose values are between 0 and 1. The algorithm outputs a set $C = \{\tilde{S}(u,v): u,v \in V, u\not= v\}$, where $\tilde{S}(u,v)$ is the $(\epsilon, \delta)$- approximation of cosine similarity between $u$ and $v$, i.e., with probability at least $1-\delta$, the worst approximation error in $C$ is at most $\epsilon$. Each vector in $V$ contains $m$ features. We take the $m$ features as the sample space $D = \{1,\cdots,m\}$. For each feature $s\in D$, let $f_{u,v}(s) = u_s\cdot v_s$, where $u_s$ is the $s$-th feature value of the vector $u$. Let $F = \{f_{u,v}: u,v\in V, u\not= v\}$. Thus $|F| < n^2/2$ since symmetry of cosine similarity. It is clear that the true average of $f_{u,v}$ equals to the cosine similarity between $u$ and $v$. Given the sampled features, the upper bound of $R_S(F)$ is 
$$R_S(F) \leq \frac{\ell}{k}\sqrt{8\log |F|} < \frac{4\ell\sqrt{\log n}}{k},$$
where $\ell = \max_{f\in F} \sum_{i=1}^k f(s_i)^2$. Since $F$ is finite, we can replace supreme by maximum. 

The approximation algorithm works in an iterative mode. For each iteration, we sample some new features $s_i$'s among $D$ and aggregate the feature products given by $f_{u,v}(s_i)$ for each $f_{u,v}$ in $F$. Then we compute the error upper bound:
$$\Delta = \frac{4\ell\sqrt{\log n}}{k} +\left(m+m\sqrt{\frac{8}{k}\log \frac{2}{\delta}} + m\sqrt{\frac{8}{k}\log \frac{2}{\delta} + \frac{4\ell\sqrt{\log n}}{k}}\right)\sqrt{\frac{\log \frac{8}{\delta}}{2k}},$$
where $\ell = \max_{f\in F} \sum_{i=1}^k f(s_i)^2$, and $k$ is the number of aggregated samples. 
If $\Delta \leq \epsilon$, then we stop and return the averages $\frac{1}{k}\sum_{s_i\in S}f_{u,v}(s_i)$ for each pair $u, v$. Otherwise, we continue to the next round, where we will sample more features. If $\Delta \leq \epsilon$ can never be satisfied, at the end we will sample all the $m$ features and return the averages as the exact solution. Algorithm \ref{alg:csa} gives the pseudocode of our solution.

It is clear to verify the correctness of our algorithm by Theorem \ref{thm:main} and Theorem \ref{thm2}.

\begin{algorithm}[!t]
\caption{\textsf{Cosine Similarity Approximation}}
\label{alg:csa}
\renewcommand{\algorithmicrequire}{\textbf{Input:}}
\renewcommand{\algorithmicensure}{\textbf{Output:}}
\begin{algorithmic}
\Require{vectors $V$ ($|V|=n$); features $U$ ($|U|=m$); $\epsilon \in (0,1)$; $\delta \in (0,1)$.}
\Ensure{$Sim(u,v)$ for each $u, v \in V$ s.t. $u\not= v$.}
\State $k \gets 0$;
\State $S \gets \emptyset$;
\State $S(u,v) \gets 0$ for each $u, v \in V$ s.t. $u\not= v$;
\While {$k \leq m$}
	\State $k' \gets \mathsf{next-sample-size}(k)$;
	\For {$i$ from $k$ to $k'-1$}
		\State Uniformly sample a feature $s$ from $U\setminus S$;
		\State $S(u,v) \gets S(u,v)+u[s]\cdot v[s]$;
		\State $S \gets S\cup \{s\}$;
	\EndFor
	\State $\Delta \gets \frac{4\ell\sqrt{\log n}}{k} +\left(m+m\sqrt{\frac{8}{k}\log \frac{2}{\delta}} + m\sqrt{\frac{8}{k}\log \frac{2}{\delta} + \frac{4\ell\sqrt{\log n}}{k}}\right)\sqrt{\frac{\log \frac{8}{\delta}}{2k}}$;
	\If {$\Delta \leq \epsilon$}
		\State break the while loop;
	\EndIf
\EndWhile
\State $\mathsf{S} \gets \{S(u,v)/k : u,v\in V, u\not=v\}$.
\State {\bf return} $\mathsf{S}$.
\end{algorithmic}
\end{algorithm}

\section{Approximating SimRank}
We approximate SimRank scores on the nodes in the digraph $G=(V,E)$. 
The basic idea is similar to the case of cosine similarity approximation, but the sample space $D$ is changed to the $2T$-dimensional manifold $[0,1]^{2T}$, where $[0,1]$ is the interval of real numbers between 0 and 1, and $F$ is now defined as
$$F = \{f_{a,b} : a,b\in V, a\not= b\},$$
where given any two nodes $a$, $b$, the function $f_{a,b}$ takes a sample $s_i\in D$ as input and outputs a value in the interval $[0,c]$. Algorithm \ref{alg:rwg} shows how to generate a random walk given a sample from $[0,1]^T$ and a starting node $a$. It is clear that the random walk chosen by Algorithm \ref{alg:rwg} is uniformly distributed if the sample is uniformly drawn from $[0,1]^T$. Thus given two nodes $a$, $b$ and a sample $s_i \in [0,1]^{2T}$, we can generate two random walks $W_a$, $W_b$ starting from $a$ and $b$ respectively. Let 
$$l_{a,b}(s_i) = \min\{j: 1\leq j\leq T, W_a[j] = W_b[j]\},$$
i.e., the number of steps before two surfers starting from $a$, $b$ meet at some node. If $W_a$ and $W_b$ never meet within $T$ steps, $l_{a,b}(s_i)$ is defined to be 0.

\begin{algorithm}[!t]
\caption{\textsf{Random Walk Generation}}
\label{alg:rwg}
\renewcommand{\algorithmicrequire}{\textbf{Input:}}
\renewcommand{\algorithmicensure}{\textbf{Output:}}
\begin{algorithmic}
\Require{reversed graph $\overline{G}$, sequence $(s_1,\cdots,s_T) \in [0,1]^T$, starting node $a$.}
\Ensure{a random walk $W = (a,\cdots)$ starting from $a$.}
\State Let $W$ be a sequence of length $T+1$;
\State $W[0] \gets a$;
\For {$i$ from 1 to $T$}
	\State $I \gets \{b: \textrm{there is an edge in $\overline{G}$ from $W[i-1]$ to $b$}\}$;
	\State $k \gets |I|$;
	\If{k=0}
		\State $W[i] \gets W[i-1]$;
	\Else
		\State Sort $I$ in lexicographical order: $(I[0],\cdots,I[k-1])$;
		\State $x \gets \lfloor s_{i-1} * k \rfloor$;
		\State $W[i] \gets I[x]$;
	\EndIf
\EndFor
\State {\bf return} $W$.
\end{algorithmic}
\end{algorithm}

The function $f_{a,b}(s_i)$ is defined as 
$$f_{a,b}(s_i) =\begin{cases}
0 & \textrm{if $l_{a,b}(s_i)=0$,} \\
c^{l_{a,b}(s_i)} & \textrm{otherwise}.\\
\end{cases}$$
It is clear that the true average $A_D(f_{a,b})$ equals to the SimRank between the nodes $a$ and $b$. Thus a similar approximation idea based on Rademacher average can be applied. We still need to upper bound the worst approximation error. The new bound is stated as follows.

\begin{theorem}
\label{thm:new}
Let $F$ be a collection of functions $f$ mapping $D$ to $[0,c]$. 
With probability at least $1-\delta$, we have
$$\sup_{f\in F}|A_S(f) - A_D(f)| \leq R_S(F) + \left(c+c\sqrt{\frac{8}{k}\log \frac{2}{\delta}} + c\sqrt{\frac{8}{k}\log \frac{2}{\delta} + R_S(F)}\right)\sqrt{\frac{\log \frac{8}{\delta}}{2k}}.$$
\end{theorem}
The proof is essentially the same as the proof of Theorem \ref{thm:main}. We leave the discussions to the Appendix. Also the upper bound of $R_S(F)$ given by Theorem \ref{thm2} remains unchanged.

Our approximation algorithm works as follows. For each iteration, we sample some new features $s_i$'s among $D$, generate random walks $W_a$, $W_b$ from nodes $a$, $b$ given $s_i$ and aggregate $f_{a,b}(s_i)$ for each $f_{a,b}$ in $F$. Then we compute the error upper bound:
$$\Delta = \frac{4\ell\sqrt{\log n}}{k} +\left(c+c\sqrt{\frac{8}{k}\log \frac{2}{\delta}} + c\sqrt{\frac{8}{k}\log \frac{2}{\delta} + \frac{4\ell\sqrt{\log n}}{k}}\right)\sqrt{\frac{\log \frac{8}{\delta}}{2k}},$$
where $\ell = \max_{f\in F} \sum_{i=1}^k f(s_i)^2$, and $k$ is the aggregated samples. 
If $\Delta \leq \epsilon$, then we stop and return the averages $\frac{1}{k}\sum_{s_i\in S}f_{a,b}(s_i)$ for each pair $u, v$. Otherwise, we continue to the next round, where we will sample more features. If $\Delta \leq \epsilon$ can never be satisfied with a certain number of iterations, we return the approximated SimRank scores of node pairs, with the message that the input parameters $(\epsilon,\delta)$ cannot be satisfied by our algorithm.

It is clear to verify the correctness of our algorithm by Theorem \ref{thm:new} and Theorem \ref{thm2}.

\begin{algorithm}[!t]
\caption{\textsf{SimRank Approximation}}
\label{alg:sra}
\renewcommand{\algorithmicrequire}{\textbf{Input:}}
\renewcommand{\algorithmicensure}{\textbf{Output:}}
\begin{algorithmic}
\Require{reversed graph $\overline{G}$; $\epsilon \in (0,1)$; $\delta \in (0,1)$; maximum random walk length $T$; maximum number of iterations $R$; SimRank constant $c$.}
\Ensure{$\mathsf{SimRank}(u,v)$ for each $u, v \in V$ s.t. $u\not= v$.}
\State $k \gets 0$;
\State $S(u,v) \gets 0$ for each $u, v \in V$ s.t. $u\not= v$;
\For {$\mathsf{count}$ from 1 to $R$}
	\State $k' \gets \mathsf{next-sample-size}(k)$;
	\For {$i$ from $k$ to $k'-1$}
		\State Uniformly sample a feature $s$ from $[0,1]^{2T}$;
		\For {each $a,b \in V\times V$ s.t. $a\not= b$}
			\State $W_a \gets \textsf{Random Walk Generation}(s[1\cdots T])$;
			\State $W_b \gets \textsf{Random Walk Generation}(s[(T+1)\cdots 2T])$;
			\State $L \gets \{j: 1\leq j\leq T, W_a[j] = W_b[j]\}$;
			\If {$|L| > 0$}
				\State $l_{a,b} \gets \min L$;
				\State $S(u,v) \gets S(u,v)+c^{l_{a,b}}$;
			\EndIf
		\EndFor
	\EndFor
	\State $\Delta \gets \frac{4\ell\sqrt{\log n}}{k} +\left(c+c\sqrt{\frac{8}{k}\log \frac{2}{\delta}} + c\sqrt{\frac{8}{k}\log \frac{2}{\delta} + \frac{4\ell\sqrt{\log n}}{k}}\right)\sqrt{\frac{\log \frac{8}{\delta}}{2k}}$;
	\If {$\Delta \leq \epsilon$}
		\State break the while loop;
	\EndIf
\EndFor
\State $\mathsf{S} \gets \{S(u,v)/k : u,v\in V, u\not=v\}$.
\If {$\mathsf{count} = R$}
	\State show a message that $(\epsilon,\delta)$ cannot be satisfied.
\EndIf
\State {\bf return} $\mathsf{S}$.
\end{algorithmic}
\end{algorithm}

%\bibliographystyle{./IEEEtran}
\bibliographystyle{plain}
\bibliography{./cos}

\section{Appendix}
\subsection{Proof of Theorem \ref{thm:finite}}
First, for any $\epsilon > 0$,
$$\begin{aligned}
\Pr[U_S(F) > \epsilon] =& \Pr[\exists f\in F, A_S(f) - A_D(f) > \epsilon] \\
\leq & \sum_{f\in F} \Pr[A_S(f) - A_D(f) > \epsilon] \\
=& \sum_{f\in F} \Pr[\frac{1}{k} \sum_{i=1}^k f(s_i) - A_D(f) > \epsilon]. \\
\end{aligned}$$ 
By Hoeffding's inequality \cite{BBL04} we have
$$\Pr[\frac{1}{k} \sum_{i=1}^k f(s_i) - A_D(f) > \epsilon] \leq \exp\left(-\frac{2k\epsilon^2}{m^2}\right).$$
Hence
$$\Pr[U_S(F) > \epsilon] \leq |F| \exp\left(-\frac{2k\epsilon^2}{m^2}\right).$$
Similarly one can show that
$$\Pr[\sup_{f\in F} [ A_D(f) - A_S(f) ] > \epsilon] \leq |F| \exp\left(-\frac{2k\epsilon^2}{m^2}\right).$$
Thus putting the two cases together,
$$\Pr[\sup_{f\in F} | A_S(f) - A_D(f) | > \epsilon] \leq 2|F| \exp\left(-\frac{2k\epsilon^2}{m^2}\right).$$
Equivalently for any $\delta>0$, with probability at least $1-\delta$,
$$\sup_{f\in F} | A_S(f) - A_D(f) | \leq m\sqrt{\frac{\log(2|F|) + \log(1/\delta)}{2k}}.$$
We have obtained the desired upper bound in Theorem \ref{thm:finite}.

\subsection{Proof of Theorem \ref{thm:main}}
In this section, we show our main result (Theorem \ref{thm:main}).
We start from the definition of self bounding function \cite{Oneto13}.
\begin{definition}
Let $s_1$, $s_2$, $\cdots$, $s_k$ be independent random variables taking values from a set $D$. A function $f: D^k \to [0, +\infty]$ is a self bounding function if there exists a constant $c$ and a function $g: D^{k-1}\to \mathbb{R}$ such that for any $s_1$, $\cdots$, $s_{j-1}$, $s_{j+1}$, $\cdots$, $s_k \in D$, the following conditions hold:
$$0 \leq f(s_1, \cdots, s_k) - g(s_1,\cdots, s_{j-1}, s_{j+1}, \cdots, s_k) \leq c,$$
$$\sum_{j=1}^k [f(s_1, \cdots, s_k) - g(s_1,\cdots, s_{j-1}, s_{j+1}, \cdots, s_k)] \leq f(s_1, \cdots, s_k).$$
\end{definition}

The following concentration inequality can be achieved for self bounding functions \cite{BLM99}.

\begin{lemma}
\cite{BLM99} If a function $Z = f(s_1,\cdots,s_k)$ is a self bounding function with constant $c$, then for $t \leq \E Z$,
$$\Pr[\E Z - Z \geq t] \leq \exp\left(-\frac{t^2}{2c\E Z}\right).$$
For $t > \E Z$, the left probability is zero trivially.
Here we take randomness over $s_1$, $s_2$, $\cdots$, $s_k$.
\end{lemma}

By using the above lemma, we can show a similar inequality for Rademacher average. 

\begin{lemma}
\label{lem1}
$$\Pr[\E R_S(F) \geq R_S(F) + t] \leq \exp\left(-\frac{kt^2}{4m\E R_S(F)}\right),$$
where $\E$ takes randomness over the samplings $s_1$, $s_2$, $\cdots$, $s_k$.
\end{lemma}
\begin{proof}
It suffices to show that $R_S(F)$ is a self bounding function with constant $c=2m/k$.
Define
$$Z = R_S(F) = \E_\sigma \sup_{f\in F} \left[\frac{2}{k}\sum_{i=1}^k \sigma_i f(s_i) \right],$$
$$G_j = \E_\sigma \sup_{f\in F} \left[\frac{2}{k}\sum_{i\not= j} \sigma_i f(s_i) \right].$$
It is clear that $Z$ is non-negative:
$$Z \geq \sup_{f\in F} \left[ \E_\sigma \frac{2}{k}\sum_{i=1}^k \sigma_i f(s_i) \right]=0.$$
Also it is clear that $Z \geq G_j$ for each $j$: suppose $\tilde{f}$ achieves the supreme of $G_j$. Then
$$\begin{aligned}
G_j &= \E_\sigma \left[\frac{2}{k}\sum_{i=1}^k \sigma_i \tilde{f}(s_i) - \frac{2}{k} \sigma_j \tilde{f}(s_j)\right] \\
&= \E_\sigma \left[\frac{2}{k}\sum_{i=1}^k \sigma_i \tilde{f}(s_i)\right] - \E_\sigma \left[ \frac{2}{k} \sigma_j \tilde{f}(s_j)\right] \\
&= \E_\sigma \left[\frac{2}{k}\sum_{i=1}^k \sigma_i \tilde{f}(s_i) \right]  \leq Z.
\end{aligned}$$
Next we show $Z- G_j \leq 2m/k =c$:
$$\begin{aligned}
G_j &= \E_\sigma \sup_{f\in F} \left[ \frac{2}{k} \sum_{i=1}^k \sigma_i f(s_i) - \frac{2}{k} \sigma_j f(s_j) \right] \\
&\geq \E_\sigma \sup_{f\in F} \left[ \frac{2}{k} \sum_{i=1}^k \sigma_i f(s_i) \right] - \E_\sigma \sup_{f\in F} \left[ \frac{2}{k} \sigma_j f(s_j) \right] \\
&\geq \E_\sigma \sup_{f\in F} \left[ \frac{2}{k} \sum_{i=1}^k \sigma_i f(s_i) \right] - \frac{2m}{k}.
\end{aligned}$$
Finally we need to verify $\sum_{j=1}^k Z-G_j \leq Z$:
$$\begin{aligned}
\sum_{j=1}^k G_j &= \E_\sigma \sum_{j=1}^k \sup_{f\in F} \left[ \frac{2}{k}\sum_{i\not= j} \sigma_i f(s_i) \right]\\
&\geq \E_\sigma \sup_{f\in F}\left[ \frac{2}{k} \sum_{j=1}^k \sum_{i\not= j} \sigma_i f(s_i) \right]\\
&=\frac{2(k-1)}{k} \E_\sigma \sup_{f\in F}\left[\sum_{j=1}^k \sigma_i f(s_i) \right] = (k-1)Z.
\end{aligned}$$
\end{proof}

We still need the following lemma on the relation between uniform deviation and Rademacher average.

\begin{lemma}
\label{lem2}
$$\E \sup_{f\in F} [ A_S(f) - A_D(f) ] \leq \E R_S(F),$$
$$\E \sup_{f\in F} [ A_D(f) - A_F(f) ] \leq \E R_S(F).$$
Here we take randomness over the $k$ samplings.
\end{lemma}
\begin{proof}
The proof idea is based on ghost samplings, i.e., independently draw another $k$ samples: $s_1'$, $\cdots$, $s_k'$, and then we have
$$A_D(f) = \frac{1}{m}\sum_{i=1}^m f(i) = \E \frac{1}{k} \sum_{j=1}^k f(s_j'),$$
where $\E$ takes randomness over the $k$ ghost samples.
Thus
$$\begin{aligned}
\E \sup_{f\in F} [ A_S(f) - A_D(f) ] &= \E \sup_{f\in F} \left[\frac{1}{k} \sum_{i=1}^k f(s_i) - \E \frac{1}{k} \sum_{j=1}^k f(s_j') \right]\\
&\leq \E \sup_{f\in F} \left[\frac{1}{k} \sum_{i=1}^k f(s_i) - \frac{1}{k} \sum_{j=1}^k f(s_j') \right].\\
\end{aligned}$$
Since all the samples $s$, $s'$ are independently identically distributed, flipping the sign of $f(s_i) - f(s_i')$ will not change the expected supreme, i.e.,
$$\E \sup_{f\in F} \left[\frac{1}{k} \sum_{i=1}^k f(s_i) - \frac{1}{k} \sum_{j=1}^k f(s_j') \right] = \E \sup_{f\in F} \frac{1}{k} \sum_{i=1}^k \left[\sigma_i (f(s_i) - f(s_i')) \right],$$
where $\sigma_i$ is uniformly distributed over $\{-1, 1\}$.
Since 
$$\E \sup_{f\in F} \frac{1}{k} \sum_{i=1}^k \left[\sigma_i (f(s_i) - f(s_i')) \right] \leq 2 \E \sup_{f\in F} \frac{1}{k} \sum_{i=1}^k \sigma_i f(s_i)  = \E R_S(F),$$
we have shown the first inequality. The second inequality is analogous.
\end{proof}

We also need McDiarmid's inequality \cite{M89}.
\begin{lemma}
\cite{M89} Let $s_1$, $\cdots$, $s_k$ be independent random variables taking values from a set $D$. Suppose a function $h: D^k \to \mathbb{R}$ satisfies
$$\sup_{x_1,\cdots,x_k,x_i'\in D} |h(x_1,\cdots,x_k) - h(x_1,\cdots,x_{i-1},x_i',x_{i+1},\cdots,x_k)| \leq c_i$$
for some constants $c_i$ and every $1\leq i \leq k$. Then for any $t>0$, we have
$$\Pr[h(s_1,\cdots,s_k) - \E h(s_1,\cdots,s_k) \geq t] \leq \exp\left(-\frac{2t^2}{\sum_{i=1}^k c_i^2}\right).$$
\end{lemma}

By the above three lemmas, we can bound the difference between true average and sampled average as follows.
\begin{lemma}
$$\begin{aligned}
&\Pr\left[\sup_{f\in F} |A_D(f) - A_S(f)| \geq R_S(F) + t\right] \\
\leq& 4\exp\left(-\frac{2kt^2}{(m+\sqrt{8m\E R_S(F)})^2} \right).\end{aligned}$$
\end{lemma}
\begin{proof}
First by Lemma \ref{lem2},
$$\begin{aligned}
&\Pr\left[\sup_{f\in F} [A_D(f) - A_S(f)] \geq R_S(F) + t\right] \\
\leq& \Pr\left[\sup_{f\in F} [A_D(f) - A_S(f)] \geq \E\sup_{f\in F} [A_D(f) - A_S(f)] + at\right] \\
&+ \Pr\left[\E R_S(F) \geq R_S(F)+(1-a)t\right] \\
\end{aligned}$$
for any $a\in [0,1]$. Let
$$h(s_1,\cdots,s_k) = A_D(f) - A_S(f) = A_D(f) - \frac{1}{k}\sum_{i=1}^k f(s_i).$$
It is clear that 
$$\begin{aligned}
&\sup_{x_1,\cdots,x_k,x_i'\in D} |h(x_1,\cdots,x_k) - h(x_1,\cdots,x_{i-1},x_i',x_{i+1},\cdots,x_k)| \\
=&\sup_{x_1,\cdots,x_k,x_i'\in D} \left|\frac{1}{k}\sum_{j=1, j\not=i}^k f(x_j)+\frac{1}{k}f(x_i') - \frac{1}{k}\sum_{i=1}^k f(x_i)\right| \\
=&\sup_{x_1,\cdots,x_k,x_i'\in D} \left|\frac{1}{k}f(x_i') - \frac{1}{k} f(x_i)\right| \leq \frac{m}{k}.
\end{aligned}$$
By McDiarmid's inequality,
$$\begin{aligned}
&\Pr\left[\sup_{f\in F} [A_D(f) - A_S(f)] \geq \E\sup_{f\in F} [A_D(f) - A_S(f)] + at\right] \\
\leq& \exp\left(-\frac{2a^2t^2}{\sum_{i=1}^k m^2/k^2}\right)
=\exp\left(-\frac{2ka^2t^2}{m^2}\right).
\end{aligned}$$

By Lemma \ref{lem1},
$$\Pr\left[\E R_S(F) \geq R_S(F)+(1-a)t\right] \leq \exp\left(-\frac{k(1-a)^2t^2}{4m\E R_S(F)}\right).$$
Let $a = 1/(1+\sqrt{8\E R_S(F) / m})$. Then putting everything together, we have 
$$\begin{aligned}
&\Pr\left[\sup_{f\in F} [A_D(f) - A_S(f)] \geq R_S(F) + t\right] \\
\leq& 2\exp\left(-\frac{2kt^2}{(m+\sqrt{8m\E R_S(F)})^2} \right).\end{aligned}$$
Similarly one can show 
$$\begin{aligned}
&\Pr\left[\sup_{f\in F} [A_S(f) - A_D(f)] \geq R_S(F) + t\right] \\
\leq& 2\exp\left(-\frac{2kt^2}{(m+\sqrt{8m\E R_S(F)})^2} \right).\end{aligned}$$
Thus we have the inequality as desired.
\end{proof}
By the above lemma we have the following important corollary.
\begin{corollary}
With probability at least $1-\delta$, we have
$$\sup_{f\in F}|A_S(f) - A_D(f)| \leq R_S(F) + (m+\sqrt{8m \E R_S(F)})\sqrt{\frac{\log \frac{4}{\delta}}{2k}}.$$
\end{corollary}

We still need to upper bound $\E R_S(F)$. By Lemma \ref{lem1}, with probability at least $1-\delta$,
$$\E R_S(F) \leq R_S(F) + \sqrt{4m\E R_S(F)\frac{\log \frac{1}{\delta}}{k}}.$$
Or equivalently,
$$\sqrt{\E R_S(F)} \leq \sqrt{\frac{m}{k}\log \frac{1}{\delta}} + \sqrt{\frac{m}{k}\log \frac{1}{\delta} + R_S(F)}.$$
Hence with probability at least $1-2\delta$, we have
$$\sup_{f\in F}|A_S(f) - A_D(f)| \leq R_S(F) + \left(m+m\sqrt{\frac{8}{k}\log \frac{1}{\delta}} + m\sqrt{\frac{8}{k}\log \frac{1}{\delta} + \frac{8R_S(F)}{m}}\right)\sqrt{\frac{\log \frac{4}{\delta}}{2k}}.$$
We have shown Theorem \ref{thm:main}.


\subsection{Proof of Theorem \ref{thm2}}
For any $s>0$, by Jensen's inequality,
$$\begin{aligned}
\exp(skR_S(F)) & = \exp\left(2s\E_\sigma \sup_{f\in F} \sum_{i=1}^k \sigma_i f(s_i)\right) \\
&\leq \E_\sigma \exp\left(2s \sup_{f\in F} \sum_{i=1}^k \sigma_i f(s_i)\right) \\
&\leq \E_\sigma \sum_{f\in F} \exp\left(2s\sum_{i=1}^k \sigma_i f(s_i)\right).
\end{aligned}$$
By Hoeffding's Lemma \cite{H63},
$$\begin{aligned}
&\E_\sigma \sum_{f\in F} \exp\left(2s\sum_{i=1}^k \sigma_i f(s_i)\right) \\
\leq & \sum_{f\in F}\prod_{i=1}^k \exp\left(2s^2f(s_i)^2\right) \\
=& \sum_{f\in F} \exp\left(2s^2\sum_{i=1}^k f(s_i)^2\right).
\end{aligned}$$
Let $\ell^2 = \sup_{f\in F}\sum_{i=1}^k f(s_i)^2$, and then
$$\sum_{f\in F} \exp\left(2s^2\sum_{i=1}^k f(s_i)^2\right) \leq |F| \exp\left(2s^2\ell^2\right).$$
Thus
$$R_S(F) \leq \frac{1}{sk}(\log|F| + 2s^2\ell^2),$$
for any $s>0$. It turns out that to minimize the right hand side of the above equation, we have
$$s = \sqrt{\frac{\log|F|}{2\ell^2}}.$$
Then
$$R_S(F) \leq \frac{\ell}{k}\sqrt{8\log |F|}.$$ 


\subsection{Discussions of Theorem \ref{thm:new}}
Theorem \ref{thm:new} can be treated as a special case of Theorem \ref{thm:finite} and Theorem \ref{thm:main} when $m=c$. All the reasoning will not be affected and thus the desired bound follows.
\end{document}
